\documentclass[12pt]{article}
\usepackage[printonlyused,withpage]{acronym} % acronyms
\usepackage{amsmath}
\usepackage{amssymb}
\usepackage{amsthm}
\usepackage[page,title,toc]{appendix} % appendix
\usepackage{array} % tables (?)
\usepackage{bbold} % \mathbb
\usepackage[backend=biber, style=apa]{biblatex} % bibliography
\usepackage{bm} % bold math
\usepackage{caption} % better captioning and figure numbering (?)
\usepackage{changepage} % some landscape pages
\usepackage{csquotes} % \blockquote
\usepackage{enumitem} % control spacing
\usepackage{etoolbox} % for a macro (?)
\usepackage{fancyhdr}				% For fancy header/footer
\usepackage{float}      % use the [H] option for positioning
\usepackage[T1]{fontenc}
\usepackage{fontspec}
\usepackage[hang,flushmargin]{footmisc} % don't indent footnotes
\usepackage{gensymb} % \degree symbol
\usepackage{glossaries} % glossary
\usepackage{graphicx}				% For including figure/image
\usepackage[margin=1.0in,includehead,includefoot]{geometry}		% For setting margins
\usepackage{import}    % \incfig
\usepackage[iso,american]{isodate}
\usepackage{makecell}  % multi-row table cells
%\usepackage{mathptmx}  % make math Times New Roman
%\usepackage{minted}    % code styling
\usepackage{pdflscape} % make \incfig to landscape
\usepackage{pdfpages} % \includepdf
%\usepackage{pgffor} % iterate over the modular files to import their labels
\usepackage{pgfplots} % bar chart
\usepackage{ragged2e} % \raggedright
\usepackage{setspace} % {single,one-half}space \doublespacing
\usepackage{soul} % underline across line breaks
\usepackage{svg}
\usepackage{textcomp} % textcomp to get rid of gensymb warnings
%\usepackage{times} % Times New Roman, only pdflatex
\usepackage{transparent} % \incfig (breaks pgffor)
\usepackage[normalem]{ulem} % allow \smash of underlines
\usepackage{vcell} % vertical writing in table cells (?)
\usepackage{wrapfig} % \incfig
\usepackage{xifthen} % \incfig
\usepackage{xparse} % ?
\usepackage{xr} % ?

% hYpErReF mUsT cOmE lAsT
\usepackage[hidelinks]{hyperref} % \href{url}{text}
\hypersetup{colorlinks=true,linkcolor=black,urlcolor=blue,filecolor=black,citecolor=black}

% svg and Inkscape's pdf_tex
\newcommand{\incfig}[2]{
	\def\svgwidth{#1\columnwidth}
	\import{./}{#2.pdf_tex}
}

% Bibliography
%\addbibresource{}

% Font size reminder
% \tiny \small \normalsize \large \Large \Huge
%\setmainfont{Times New Roman}
\setmonofont{Mononoki Nerd Font Mono}[Scale=MatchLowercase]

%%% Macros
% Horizontal Spacing
\newcommand\hs{\hspace{1cm}}
\newcommand\hhs{\hspace{0.5cm}}
% Singlespacing in certain environments: \ssInEnv{environment, font=\normalsize}
% \newcommand{\ssInEnv}[2][\normalsize]{\BeforeBeginEnvironment{#2}{\begin{singlespace*}{#1}}\AfterEndEnvironment{#2}{\end{singlespace*}}}
% Fancy header and footer for homeworks
\newcommand\hwheadfoot{
\pagestyle{fancy}
\fancyhead[LO,L]{Lorenzo Hess}
\fancyhead[CO,C]{CLASS: Homework NUMBER} % TODO
\fancyhead[RO,R]{\today} % TODO
%\fancyfoot[LO,L]{}
\fancyfoot[CO,C]{\thepage}
%\fancyfoot[RO,R]{}
\topmargin=-0.75in
\renewcommand{\headrulewidth}{0.4pt}
\renewcommand{\footrulewidth}{0.4pt}}

% Homework problems with arbitrary numbers without worrying about section formatting and counters
\newcommand{\problemSection}[1]{ % Args = section number
\noindent\Large\textbf{Section #1}
}
\newcommand{\problem}[1]{ % Args = number
\noindent\large\textbf{Problem #1}
}

% Volume symbol (heat transfer)
\newcommand{\volume}{{\ooalign{\hfil$V$\hfil\cr\kern0.08em--\hfil\cr}}}

% Table of Contents tweaks
% Singlespacing
\let\oldToC\tableofcontents
\renewcommand{\tableofcontents}{\begin{singlespace}\oldToC\end{singlespace}}
% Include subsubsections
\setcounter{tocdepth}{3}

% Spacing
\singlespacing
% \ssInEnv{itemize}
% \ssInEnv{enumerate}
% \ssInEnv{tabular}
\setlist{listparindent=\parindent, % indent paragraphs under \item
  nolistsep} % no spacing between list items

% Smash underlines
\let\oldunderline\underline
\renewcommand{\underline}[1]{\oldunderline{\smash{#1}}}
\setlength\ULdepth{1.5pt}

% Minted code styling
%\usemintedstyle{friendly}
%\setminted[matlab]{fontsize=\normalsize, breaklines=true, breakanywhere=true, linenos=true, numbersep=6pt, stripnl=true, baselinestretch=1}
%\setminted[text]{fontsize=\scriptsize, breaklines=true, breakanywhere=true, linenos=false, numbersep=6pt, stripnl=true, baselinestretch=1}

\author{Lorenzo Hess}
\date{\today}
\title{Homework 1: Linkage Analysis}

\begin{document}
\maketitle
\tableofcontents

\section{Assumptions}%
\label{assum}

\begin{enumerate}
        \item Steady-state rate of 7450 parts in seven hours
\end{enumerate}


\section{Free-Body Diagrams}%
\label{fbds}

\section{Statics and Dynamics Equations}%
\label{eqns}

\subsection{Forces and Moments}%
\label{eqns.forces-moments}

\subsubsection{Static Equilibrium}%
\label{eqns.forces-moments.static}

At static equilibrium, the net force and moment on each link is zero. Figures \ref{fig:fbd-ab}, \ref{fig:fbd-bc}, \ref{fig:fbd-cde}, \ref{fig:fbd-ef}, and \ref{fig:fbd-fg} show free-body diagrams of each link, with an input torque $\tau_{\text{in}}$, weights $W$, at center of masses $s$, and with artifact weight $W_{\text{art}}$.

Each link yields one vector force equations; link AB also yields one vector moment equation. All force equations have two components, and the moment equation has three components.

For link AB, with moments summed at point $A$:
\[ R_{a,x} - R_{b,x} = 0 \hs R_{a,y} - R_{b,y} - W_{ab} = 0 \hs \vec{\tau}_{in} - \vec{r}_{s/a}\times \vec{W}_{ab} - \vec{r}_{b/a}\times \vec{R}_{b} \]
For link BC:
\[ R_{b,x} - R_{c,x} = 0 \hs R_{b,y} - R_{c,y} - W_{bc} = 0 \]
For link CDE:
\[ R_{d,x} + R_{c,x} - R_{e,x} = 0 \hs R_{d,y} + R_{c,y} - R_{e,y} - W_{cde} = 0 \]
For link EF:
\[ R_{e,x} - R_{f,x} = 0 \hs R_{e,y} - R_{f,y} - W_{ef} = 0 \]
For link FG:
\[ R_{f,x} - R_{g,x} = 0 \hs R_{f,y} - R_{g,y} - W_{fg} - W_{\text{art}} = 0 \]

\subsubsection{Dynamic Equilibrium}%
\label{eqns.forces-moments.dynamic}

\subsection{Mass Calculations}%
\label{eqns.masses}

\subsection{Mass Moment of Inertia Calculations}%
\label{eqns.mmi}

\subsection{Kinematics Equations}%
\label{eqns.kinematics}

I developed my kinematic equations from two vector loops: A-B-C-D-A and D-E-F-G-D. Note that these represent two four-bar linkages in series, with ternary link DEC as the common link.

\subsubsection{Position}%
\label{eqns.kinematics.position}

For loop ABCDA, the position loop is:
\[ \vec{r}_{b/a} + \vec{r}_{c/b} + \vec{r}_{d/c} + \vec{r}_{a/d} = 0 \]
For loop DEFGD, the position loop is:
\[ \vec{r}_{e/d} + \vec{r}_{f/e} + \vec{r}_{g/f} + \vec{r}_{g/d} = 0 \]

\subsubsection{Velocity}%
\label{eqns.kinematics.velocity}

Differentiate the positions loop equations to derive the velocity loop equations:
\[ \vec{v}_{b/a} + \vec{v}_{c/b} + \vec{v}_{d/c} + \vec{v}_{a/d} = 0 \]
\[ \vec{v}_{e/d} + \vec{v}_{f/e} + \vec{v}_{g/f} + \vec{v}_{g/d} = 0 \]
The velocity of a joint $j$ relative to $i$ can be decomposed into translational and rotational components:
\[ \vec{v}_{j/i} = \vec{v}_{i} + (\omega_{ij}\times \vec{r}_{j/i}) \]
For a ground joint $i$, $\vec{v}_{i}$ is zero. For joints $i$ and $j$ of the same rigid link, their relative translational velocity is also zero. All joints and links in this linkage satisfy these conditions, so all translational velocity terms are zero for all joints. Therefore, the velocity loop equations for the loop equations reduce to the rotational components:
\[ (\vec{\omega}_{ab}\times \vec{r}_{b/a}) + (\vec{\omega}_{bc}\times \vec{r}_{c/b}) + (\vec{\omega}_{cd}\times \vec{r}_{d/c}) + (\vec{\omega}_{da}\times \vec{r}_{a/d}) = 0 \]
\[ (\vec{\omega}_{de}\times \vec{r}_{e/d}) + (\vec{\omega}_{ef}\times \vec{r}_{f/e}) + (\vec{\omega}_{fg}\times \vec{r}_{g/f}) + (\vec{\omega}_{dg}\times \vec{r}_{g/d}) = 0 \]\\

To solve the loop equations we require the steady-state crank angular velocity, $\omega_{1}$. With a part per hour rate of 7450 parts per seven hours, we can compute:
\[ \dot{p} = \frac{7450 \text{part}}{7 \text{hr}} = 0.29 \frac{\text{part}}{\text{sec}} \]
\[ p^{-1} = 3.38 \frac{\text{s}}{\text{part}} \]
By inspection of the PMKS+ model, we see that the output link completes one cycle at the same rate as the input link. Thus:
\[ \omega_{1} = \omega_{5} \]
The output link delivers one part per revolution, so the output link angular velocity is:
\[ \omega_{5} = \frac{2\pi}{p^{-1}}\cdot \text{1 part}  \]
\[ \Rightarrow \omega_{1} = 1.86 \text{rad}/\text{s} \]

\subsubsection{Acceleration}%
\label{eqns.kinematics.acceleration}

Differentiate the velocity loop equations to derive the acceleration loop equations. Begin by differentiating the general rotational velocity:
\[ \frac{d }{d t} \left( \omega_{j/i}\times \vec{r}_{j/i} \right) = (\vec{\alpha}_{j/i}\times \vec{r}_{j/i}) +  \omega_{j/i}\times (\omega_{j/i}\times \vec{r}_{j/i}) \]
Written compactly, the acceleration loop equations are thus:
\[ \Sigma (\vec{\alpha}_{j/i}\times \vec{r}_{j/i}) +  \omega_{j/i}\times (\omega_{j/i}\times \vec{r}_{j/i}) \text{ for } i,j \cap \{(a,b),(b,c),(c,d),(d,a)\} \]
\[ \Sigma (\vec{\alpha}_{j/i}\times \vec{r}_{j/i}) +  \omega_{j/i}\times (\omega_{j/i}\times \vec{r}_{j/i}) \text{ for } i,j \cap \{(d,e),(e,f),(f,g),(g,d)\} \]

\subsection{Accelerations at CMs}%
\label{eqns.accels}

The acceleration at the center of mass of a link

\section{Results}%
\label{res}

\subsection{First Position}%
\label{res.first}

\subsubsection{Joint Forces and Torques}%
\label{res.first.joints}

\subsubsection{Postion, Velocity, and Acceleration}%
\label{res.first.kin}

\subsubsection{Masses and Mass Moments of Inertia}%
\label{res.first.mass-mmi}

\subsection{Plots}%
\label{res.plots}

\subsection{Comparison with PMKS+}%
\label{res.compare}

\section{MATLAB Code}%
\label{code}

\section{Discussion}%
\label{discuss}

\section{References}%
\label{ref}

\end{document}
