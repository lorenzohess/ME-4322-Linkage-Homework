\documentclass[12pt]{article}
\usepackage[printonlyused,withpage]{acronym} % acronyms
\usepackage{amsmath}
\usepackage{amssymb}
\usepackage{amsthm}
\usepackage[page,title,toc]{appendix} % appendix
\usepackage{array} % tables (?)
\usepackage{bbold} % \mathbb
\usepackage[backend=biber, style=apa]{biblatex} % bibliography
\usepackage{bm} % bold math
\usepackage{caption} % better captioning and figure numbering (?)
\usepackage{changepage} % some landscape pages
\usepackage{csquotes} % \blockquote
\usepackage{enumitem} % control spacing
\usepackage{etoolbox} % for a macro (?)
\usepackage{fancyhdr}				% For fancy header/footer
\usepackage{float}      % use the [H] option for positioning
\usepackage[T1]{fontenc}
\usepackage{fontspec}
\usepackage[hang,flushmargin]{footmisc} % don't indent footnotes
\usepackage{gensymb} % \degree symbol
\usepackage{glossaries} % glossary
\usepackage{graphicx}				% For including figure/image
\usepackage[margin=1.0in,includehead,includefoot]{geometry}		% For setting margins
\usepackage{import}    % \incfig
\usepackage[iso,american]{isodate}
\usepackage{makecell}  % multi-row table cells
%\usepackage{mathptmx}  % make math Times New Roman
%\usepackage{minted}    % code styling
\usepackage{pdflscape} % make \incfig to landscape
\usepackage{pdfpages} % \includepdf
%\usepackage{pgffor} % iterate over the modular files to import their labels
\usepackage{pgfplots} % bar chart
\usepackage{ragged2e} % \raggedright
\usepackage{setspace} % {single,one-half}space \doublespacing
\usepackage{soul} % underline across line breaks
\usepackage{svg}
\usepackage{textcomp} % textcomp to get rid of gensymb warnings
%\usepackage{times} % Times New Roman, only pdflatex
\usepackage{transparent} % \incfig (breaks pgffor)
\usepackage[normalem]{ulem} % allow \smash of underlines
\usepackage{vcell} % vertical writing in table cells (?)
\usepackage{wrapfig} % \incfig
\usepackage{xifthen} % \incfig
\usepackage{xparse} % ?
\usepackage{xr} % ?

% hYpErReF mUsT cOmE lAsT
\usepackage[hidelinks]{hyperref} % \href{url}{text}
\hypersetup{colorlinks=true,linkcolor=black,urlcolor=blue,filecolor=black,citecolor=black}

% svg and Inkscape's pdf_tex
\newcommand{\incfig}[2]{
	\def\svgwidth{#1\columnwidth}
	\import{./}{#2.pdf_tex}
}

% Bibliography
%\addbibresource{}

% Font size reminder
% \tiny \small \normalsize \large \Large \Huge
%\setmainfont{Times New Roman}
\setmonofont{Mononoki Nerd Font Mono}[Scale=MatchLowercase]

%%% Macros
% Horizontal Spacing
\newcommand\hs{\hspace{1cm}}
\newcommand\hhs{\hspace{0.5cm}}
% Singlespacing in certain environments: \ssInEnv{environment, font=\normalsize}
% \newcommand{\ssInEnv}[2][\normalsize]{\BeforeBeginEnvironment{#2}{\begin{singlespace*}{#1}}\AfterEndEnvironment{#2}{\end{singlespace*}}}
% Fancy header and footer for homeworks
\newcommand\hwheadfoot{
\pagestyle{fancy}
\fancyhead[LO,L]{Lorenzo Hess}
\fancyhead[CO,C]{CLASS: Homework NUMBER} % TODO
\fancyhead[RO,R]{\today} % TODO
%\fancyfoot[LO,L]{}
\fancyfoot[CO,C]{\thepage}
%\fancyfoot[RO,R]{}
\topmargin=-0.75in
\renewcommand{\headrulewidth}{0.4pt}
\renewcommand{\footrulewidth}{0.4pt}}

% Homework problems with arbitrary numbers without worrying about section formatting and counters
\newcommand{\problemSection}[1]{ % Args = section number
\noindent\Large\textbf{Section #1}
}
\newcommand{\problem}[1]{ % Args = number
\noindent\large\textbf{Problem #1}
}

% Volume symbol (heat transfer)
\newcommand{\volume}{{\ooalign{\hfil$V$\hfil\cr\kern0.08em--\hfil\cr}}}

% Table of Contents tweaks
% Singlespacing
\let\oldToC\tableofcontents
\renewcommand{\tableofcontents}{\begin{singlespace}\oldToC\end{singlespace}}
% Include subsubsections
\setcounter{tocdepth}{3}

% Spacing
\singlespacing
% \ssInEnv{itemize}
% \ssInEnv{enumerate}
% \ssInEnv{tabular}
\setlist{listparindent=\parindent, % indent paragraphs under \item
  nolistsep} % no spacing between list items

% Smash underlines
\let\oldunderline\underline
\renewcommand{\underline}[1]{\oldunderline{\smash{#1}}}
\setlength\ULdepth{1.5pt}

% Minted code styling
%\usemintedstyle{friendly}
%\setminted[matlab]{fontsize=\normalsize, breaklines=true, breakanywhere=true, linenos=true, numbersep=6pt, stripnl=true, baselinestretch=1}
%\setminted[text]{fontsize=\scriptsize, breaklines=true, breakanywhere=true, linenos=false, numbersep=6pt, stripnl=true, baselinestretch=1}

\author{Lorenzo Hess}
\date{\today}
\title{Homework 1: Linkage Analysis}

\begin{document}
\maketitle
\tableofcontents

\section{Assumptions}%
\label{assum}

\begin{enumerate}
        \item Steady-state rate of 7450 parts in seven hours
\end{enumerate}


\section{Free-Body Diagrams}%
\label{fbds}

\section{Statics and Dynamics Equations}%
\label{eqns}

\subsection{Forces and Moments}%
\label{eqns.forces-moments}

\subsection{Mass Calculations}%
\label{eqns.masses}

\subsection{Mass Moment of Inertia Calculations}%
\label{eqns.mmi}

\subsection{Kinematics Equations}%
\label{eqns.kinematics}

\subsubsection{Position}%
\label{eqns.kinematics.position}

\subsubsection{Velocity}%
\label{eqns.kinematics.velocity}

To solve the angular velocities, we require the steady-state crank angular velocity, $\omega_{1}$. With a part per hour rate of 7450 parts per seven hours, we can compute:
\[ \dot{p} = \frac{7450 \text{part}}{7 \text{hr}} = 0.29 \frac{\text{part}}{\text{sec}} \]
\[ p^{-1} = 3.38 \frac{\text{s}}{\text{part}} \]
By inspection of the PMKS+ model, we see that the output link completes one cycle at the same rate as the input link. Thus:
\[ \omega_{1} = \omega_{5} \]
The output link delivers one part per revolution, so the output link angular velocity is:
\[ \omega_{5} = \frac{2\pi}{p^{-1}}\cdot \text{1 part}  \]
\[ \Rightarrow \omega_{1} = 21.25 \text{rad}/\text{s} \]

\subsubsection{Acceleration}%
\label{eqns.kinematics.acceleration}

\subsection{Accelerations at CMs}%
\label{eqns.accels}

\section{Results}%
\label{res}

\subsection{First Position}%
\label{res.first}

\subsubsection{Joint Forces and Torques}%
\label{res.first.joints}

\subsubsection{Postion, Velocity, and Acceleration}%
\label{res.first.kin}

\subsubsection{Masses and Mass Moments of Inertia}%
\label{res.first.mass-mmi}

\subsection{Plots}%
\label{res.plots}

\subsection{Comparison with PMKS+}%
\label{res.compare}

\section{MATLAB Code}%
\label{code}

\section{Discussion}%
\label{discuss}

\section{References}%
\label{ref}

\end{document}
